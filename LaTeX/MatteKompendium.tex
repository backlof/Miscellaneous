\documentclass[12pt,norsk,a4paper]{article}

\title{Matematikk Kompendium}
\author{Name}



\usepackage[utf8]{inputenc}
\usepackage[norsk,english]{babel}
\usepackage[norsk]{isodate}

\usepackage{amsmath}
\usepackage{amssymb}
\usepackage{listings}
\usepackage{graphicx}



\usepackage{parskip}

\usepackage{a4wide}
\usepackage[top=2cm]{geometry}
\usepackage{color}
\usepackage{listings}
\usepackage{graphicx}

\usepackage{polynom}

\polyset{%
  style=C,
  delims={\big(}{\big)},
  div=:
}

\addto\captionsenglish{%
  \renewcommand{\contentsname}%
    {Innhold}%
}

\definecolor{drkgreen}{rgb}{0,0.6,0}
\definecolor{mauve}{rgb}{0.58,0,0.82}

\lstset{language=Matlab, basicstyle=\footnotesize, frame=single, breaklines=true, breakatwhitespace=true, keywordstyle=\color{blue}, commentstyle=\color{drkgreen}, stringstyle=\color{mauve}}

\lstset{,breaklines=true}

\lstset{basicstyle=\scriptsize\ttfamily, frame=single, breaklines=true, breakatwhitespace=true}

\begin{document}

\maketitle

\tableofcontents

\newpage

\section{Grunnleggende}
\subsection*{Tall}
De reelle tallene $\mathbb{R}$ består av

\begin{itemize}
\item De naturlige tallene $\mathbb{N} = \{ 1,2,3 \textellipsis \}$
\item De hele tallene $\mathbb{Z} = \{ -1,0,1,2 \textellipsis \}$
\item De rasjonale tallene $\mathbb{Q} = \{ \frac{a}{b} : a,b \in \mathbb{Z}, b \neq 0 \}$
\item De irrasjonale tallene $\mathbb{R} - \mathbb{Q}$ som $\pi,\sqrt{2},e$
\end{itemize}

I tillegg finnes imaginære tall som $\sqrt{-1}$ som har egne regneregler.

\newpage









\subsection{Regneregler}
\subsubsection*{Addisjon}
\begin{align*}
a + b &= b + a		&	a - a &= 0		&	a + 0 &= a	 	\\
a+(b+c) &= (a+b)+c	&	a - b &= a+(-b)	&	a-(b+c) &= a-b-c	
\end{align*}

\subsubsection*{Multiplikasjon}
\begin{align*}
ab &= ba		&	a(bc) &= (ab)c	&	a(b + c) &= ab + ac	\\
a \cdot 1 &= a	&	a \cdot \frac{1}{a} &= 1, \quad a \neq 0 
\end{align*}

\subsubsection*{Divisjon}
\begin{align*}
\frac{a}{c} + \frac{b}{c} &= \frac{a+b}{c}		&	\frac{ac}{bc} &= \frac{a}{b}	&	\frac{a}{c} \cdot \frac{b}{d} &= \frac{ab}{cd}	&	\frac{a}{c} \div \frac{b}{d} &= \frac{a}{c} \cdot \frac{d}{b}
\end{align*}

\begin{align*}
\frac{a}{c} + \frac{b}{d} = \frac{ad}{cd} + \frac{bc}{dc} = \frac{ad + bc}{cd}
\end{align*}

\subsubsection*{Potenser}
\begin{align*}
a^{1} &= a						&	a^{0} &= 1	&	a^{-n} &= \frac{1}{a^{n}}, \quad a \neq 0	\\
\sqrt[n]{a} &= a^{\frac{1}{n}}	&	\sqrt{a+b} &\neq \sqrt{a} + \sqrt{b}
\end{align*}

\begin{align*}
a^{n} \cdot a^{m} &= a^{n+m}	&	\frac{a^{n}}{a^{m}} &= a^{n-m}				&	(a^{n})^{m} &= a^{(n \cdot m)}	\\
a^{n} \cdot b^{n} &= (ab)^{n}	&	\frac{a^{n}}{b^{n}} &= (\frac{a}{b})^{n}
\end{align*}

\subsubsection*{Store tall}
\begin{align*}
7000 &= 7 \cdot 10^{3}	&	0,72 &= 7,2 \cdot 10^{-1}
\end{align*}

\subsubsection*{Logaritmer}
Logaritmer brukes for å finne ut hvor mange ganger basen må ganges med seg selv for å bli et tall. For naturlige logaritmer er det for eksempel snakk om hvor mange ganger tallet $e = 2.71828182 \textellipsis$ må ganges med seg selv.

\begin{equation}
8 = 2 \cdot 2 \cdot 2 = 2^{3}	\quad	\rightarrow	\quad	\log_2 (8) = 3 
\end{equation}

\begin{equation}
\log_e (10) = \ln (10)
\end{equation}

Den vanlige basen for benevnelsen $\log$ er 10.

\begin{equation}
\log(10) = 1	\quad	\rightarrow	\quad	10^{1} = 10
\end{equation}

Husk at $x^0 = 1$ så lenge $x > 0$, dermed er $\log (1) = 0$ for alle baser.

\begin{align*}
\ln(1)&= 0	&	\ln (a^{b}) &= b \ln a		&	\ln (ab) &= \ln a + \ln b	&	\ln (\frac{a}{b}) &= \ln a - \ln b		\\
e^{\ln (a^{b})} &= a^{b}	&	e^{b \cdot \ln a} &= a^{b}	&	e^{\ln (ab)} &= ab		&	e^{(\ln a + \ln b)} &= ab
\end{align*}

\paragraph*{Eksempel} Hvor mange ganger må $e$ ganges seg selv for å bli 10?
\begin{align*}
e^{x} &= 10				\\
\ln (e^{x}) &= \ln (10)	\\
x &= \ln (10) = 2.30258509 \textellipsis
\end{align*}

\newpage









\subsection{Likninger}
\subsubsection{Annengradslikninger}
\begin{equation}
ax^{2} + bx + c = 0 , \quad a,b,c \in \mathbb{R} , \quad a \neq 0
\end{equation}

På diskriminanten $b^{2} - 4ac$ kan vi se hvor mange løsninger det er på svaret:

\begin{itemize}
\item[$= 0$]En løsning
\item[$> 0$]To løsninger
\item[$< 0$]Komplekse tall - ingen løsning\\
\end{itemize}

Noen ganger får man 2 reelle løsninger, men bare 1 av dem fungerer i praksis.

\textbf{Løsning 1}
\begin{equation}
x = \frac{-b \pm \sqrt{b^{2} -4ac} }{2a}
\end{equation}

\textbf{Løsning 2}
\begin{align*}
x^{2} - 2x &= 0	\\
x(x - 2) &= 0	\\
x = 0 \quad &\cup \quad x = 2
\end{align*}

\subsubsection{Rasjonale likninger}
Rasjonale likninger er likninger som inneholder rasjonale uttrykk.

\paragraph*{Eksempel} Løs likningen $\frac{3}{x - 1} = \frac{6}{x}, \quad x \neq \{ 0,1 \}$
\begin{align*}
\frac{3x(x-1)}{x-1} &= \frac{6x(x-1)}{x}				\\
3x &= 6(x - 1)											\\
-3x &= -6												\\
x &= 2
\end{align*}

\newpage








\subsubsection{Irrasjonale likninger}
Disse likningene kan man løse ved å:

\begin{enumerate}
\item Få alle rotuttrykkene alene på venstre side
\item Kvadrer på begge sider
\item Rydd og løs
\item Prøv svarene
\end{enumerate}

\paragraph*{Eksempel} Løs likningen $\sqrt{x} = 12 - x$
\begin{align*}
(\sqrt{x})^{2} &= (12 - x)^{2}	\\
x &= 144 - 24x + x^{2}			\\
x = 16 \quad &\cup \quad x = 9
\end{align*}

\subsubsection{Polynomdivisjon}
Polynomdivisjon brukes til å faktorisere uttrykk av høyere grad enn 2.

\[  \polylongdiv{x^3-6x^2+11x-6}{x-1}  \]

Det første man må gjøre ved polynomdivisjon er å finne noe det opprinnelige uttrykket kan deles på. I dette tilfellet vil det passe med $(x - 1)$ siden det er negative polynomer i uttrykket. Deretter fjerner vi polynomer fra uttrykket med denne formelen:

\begin{enumerate}
\item Hva må $x$ ganges med for å få $x^3$
\item Legg til tallet ($x^2$) etter likhetstegnet
\item Trekk fra $x^2(x - 1)$ (for å se hva som er igjen å faktorisere)
\item Neste polynom
\end{enumerate}

\newpage










\section{Derivasjon}
Derivasjon er en operasjon der en bestemmer den deriverte av en funksjon. For en funksjon av én variabel $f(x)$ er den deriverte definert ved

\begin{equation}
 f'(x) = \frac{d}{dx}(f(x)) = \lim_{h \to 0} \frac{f(x + h) - f(x)}{h}
\end{equation}

dersom grenseverdien eksisterer. Den deriverte er et mål for endringen i $f(x)$ når den frie variabelen $x$ endres, altså et uttrykk for \textit{stigningstallet}.

\subsection*{Regler}
\begin{align*}
C' &= 0							&	(x^{n})' &= n \cdot x^{(n-1)}	&	(a \cdot f(x))' &= a \cdot f'(x)	\\
(e^{x})' &= e^{x}				&	(a^{x})' &= a^{x} \cdot \ln x	&	(\ln x)' &= \frac{1}{x}				\\
(\sin x)' &= \cos x				&	(\cos x)' &= - \sin x			&	(\tan x)' &= \frac{1}{cos^{2} x}	\\
(\sin^{-1} x) ' &= \frac{1}{\sqrt{1 - x^{2}}}	&	(\cos^{-1} x) ' &= \frac{1}{\sqrt{1 - x^{2}}}	&	(\tan^{-1} x) ' &= \frac{1}{1 - x^{2}}
\end{align*}

Husk at $\sqrt{x} = x^{\frac{1}{2}}$ og dermed er $(\sqrt[m]{x^{n}})' = (x^{\frac{n}{m}})' = \frac{n}{m} \cdot x ^{(\frac{n}{m} - 1)}  $

\subsubsection*{Polynom}
\begin{equation}
(f(x) + g(x))' = f'(x) + g'(x)
\end{equation}

\subsubsection*{Produktregelen}
\begin{equation}
(f(x) \cdot g(x))' = f'(x) \cdot g(x) + f(x) \cdot g'(x)
\end{equation}

\paragraph*{Eksempel} Deriver $2x \cdot e^{x}$
\begin{align*}
f(x) &= 2x		&	g(x) &= e^{x}		\\
f'(x) &= 2 \cdot (\frac{d}{dx}(x)) = 2 \cdot 1 = 2 		&	g'(x) &= e^{x}
\end{align*}

\begin{equation*}
(2x \cdot e^{x})' = 2 \cdot e^{x} + 2x \cdot e^{x} = e^{x} (2 + 2x)
\end{equation*}

\subsubsection*{Kvotientregelen}
\begin{equation}
(\frac{g(x)}{h(x)})' = \frac{g'(x) \cdot h(x) - g(x) \cdot h'(x)}{(h(x))^{2}}
\end{equation}

\newpage

\subsection{Taylorrekker}
En taylorrekke er en representasjon av en funksjon som en rekke, der leddene er definert ved hjelp av den deriverte av funksjonen. Disse blir benyttet for å finne en tilnærming av funksjonen.

\begin{equation}
\sum\limits_{n=0}^{\infty} \frac{f^{(n)}(a)}{n!}(x - a)^{n} 
\end{equation}

\begin{equation}
f(a) + \frac{f'(a)}{1!}(x - a) + \frac{f''(a)}{2!}(x - a)^{2} + \frac{f^{(3)}(a)}{3!}(x - a)^{3} \textellipsis
\end{equation}

\paragraph*{Eksempel} Bruk et 2. ordens Taylorpolynom til å estimere $\sqrt{101}$

\begin{align*}
f(x) = \sqrt{x}, \quad a = 100
\end{align*}

\begin{align*}
f(100) &= \sqrt{100} = 10	\\
f'(x) &= \frac{1}{2\sqrt{x}} = \frac{1}{2}x^{-\frac{1}{2}}	&	f'(100) &= \frac{1}{2\sqrt{100}} = \frac{1}{20}	\\
f''(x) &= - \frac{1}{4}x^{- \frac{3}{2}} = - \frac{1}{4(\sqrt{x})^{3}}	&	f''(100) &= - \frac{1}{4 \cdot 10^3} = - \frac{1}{4000}
\end{align*}

\begin{align*}
P_{2}(x) &= f(100) + f'(100)(x - 100) + \frac{1}{2} f''(100)(x-100)^2	\\
&= 10 + \frac{1}{20} \cdot (x - 100) - \frac{1}{8000}(x - 100)^2	\\
f(101) \approx P_{2}(101) &= 10 + \frac{1}{20} \cdot (101 - 100) - \frac{1}{8000}(101 - 100)^2 = \frac{80399}{8000}
\end{align*}



\newpage













\section{Integrasjon}

\subsection*{Regler}
\begin{align*}
\int k dx &=	kx + C		&		\int x^{r} dx &= \frac{1}{1+r} \cdot x^{r+1} + C, \quad r \neq -1		\\
\int e^{x} dx &= e^{x} + C		&		\int e^{kx} dx &= \frac{1}{k} \cdot e^{kx} + C	\\
\int a^{x} dx &= \frac{1}{\ln a} \cdot a^{x} + C	&	\int a^{kx} dx &= \frac{1}{k \cdot \ln a} \cdot a^{kx} + C	\\
\int \frac{1}{x} dx &= \ln x + C, \quad x \neq 0	&	\int \ln x dx &= x \cdot \ln x - x + C
\end{align*}
\subsubsection*{Trigonometri}
\begin{align*}
\int \cos x dx &= \sin x + C	&	\int \sin x dx &= - \cos x + C	\\
\int \tan x dx &= - \ln ( \cos x) + C	&	\int \tan ^{2} x dx &= \tan x - x + C	\\
\int \frac{1}{\sqrt{1 - x^{2}}} dx &= \sin^{-1}x + C	&	\int \frac{1}{1 + x^{2}} dx &= \tan^{-1} x + C	\\
\int \frac{1}{\cos^{2} x} dx &= \tan x + C
\end{align*}

\subsection{Ubestemte integraler}

\subsubsection*{Polynom}
\begin{equation}
\int f(x) + g(x) dx = \int f(x) dx + \int g(x) dx + C
\end{equation}

\subsubsection*{Faktoriseringsregelen}
\begin{equation}
\int k \cdot f(x) dx = k \cdot \int f(x) dx + C
\end{equation}

\subsubsection*{Delvis integrasjon}
\begin{equation}
\int f(x) \cdot g'(x) dx = f(x) \cdot g(x) - \int f'(x) \cdot g(x) dx + C
\end{equation}

%*Trenger ett eksempel*

\subsubsection*{Substitusjon}
\begin{equation}
\int f(u(x)) \cdot u'(x) dx = \int f(u) du + C
\end{equation}

%*Trenger ett eksempel*

\subsection{Bestemte integraler}
Hvis $f(x)$ er kontinuerlig og $F'(x)=f(x)$ er
\begin{equation}
\int_a^b f(x) dx = [ F(x) ]_a^b = F(b) - F(a)
\end{equation}

Vi kan bytte om på grensene slik
\begin{equation}
\int_a^b f(x) dx = - \int_b^a f(x) dx
\end{equation}

\subsubsection*{Substitusjon}
\begin{equation}
\int_a^b f(u(x)) \cdot u'(x) dx = \int_{u(a)}^{u(b)} f(u) du
\end{equation}

\newpage
\section{Differensiallikninger}
Vi bruker differensiallikninger for å finne ukjente funksjoner, som regel $y(x)$. I likningen inngår $y'(x)$ eller en høyere ordens derivert. Hvis man får en initialbetingelse, f.eks $y(0) = 5$, kan vi finne $C$.

\paragraph*{Eksempel} Bestem $y(x)$ når $y'(x) = 2x + \sin x$ og $y(0) = 3$
\begin{equation*}
y(x) = \int (2x + \sin x) dx = x^2 - \cos x + C
\end{equation*}
\begin{align*}
0^2 + \cos 0 + C &= 3	\\
-1 + C &= 3				\\
C &= 4
\end{align*}
\begin{align*}
y(x) = x^2 - \cos x + 4
\end{align*}

\subsection{Lineære}
Det som kjennetegner disse likningene er at de kan skrives på formen:

\begin{equation}
y' + p(x) \cdot y = q(x)
\end{equation}

\paragraph*{Eksempel} Finn $y(x)$ for $2y' = 12 + 4y$
\begin{align*}
2y' &= 12 + 4y	\\
y' - 2y &= 6
\end{align*}

Vi ganger alle ledd med $e$ opphøyd i integralet til faktoren før $y$, $p(x)$.

\begin{align*}
\int p(x) dx = \int -2 dx = -2x + C
\end{align*}

Vi gjør dette fordi venstresiden nå følger kjerneregelen, og vi kan dermed integrere.

\begin{align*}
y' \cdot e^{-2x} - 2y \cdot e^{-2x} &= 6 \cdot e^{-2x}	\\
(y \cdot e^{-2x})' &= 6 \cdot e^{-2x}					\\
y \cdot e^{-2x} &= \int 6 \cdot e^{-2x} dx
\end{align*}

Hvis $q(x) = 0$ sier vi at likningen er \textbf{homogen}.

\newpage

\subsection{Separable}
Det som kjennetegner separable differensiallikninger er at $y$'ene og $x$'ene kan separeres på hver side av likningene.

\begin{equation}
g(y) \cdot y' = h(x)
\end{equation}

\paragraph*{Eksempel} Finn $y(x)$ for $\frac{y}{3} \cdot y' = x^2 + x$
\begin{align*}
\frac{y}{3} \cdot y' = x^2 + x
\end{align*}

Vi skriver at $y' = \frac{dy}{dx}$ og får

\begin{align*}
\int \frac{y}{3} \cdot \frac{dy}{dx} dx &= \int (x^2 + x)dx	\\
\int \frac{y}{3} dy &= \int (x^2 + x)dx
\end{align*}

Nå kan man løse differensiallikningen ved å utføre integrasjonene på hver side.












\newpage
\section{Lineære algebra}
En lineær likning består av \textit{koeffisienter}, \textit{variabler} og ett \textit{konstantledd}.

\begin{equation*}
3x + 4y = 1
\end{equation*}

Målet med lineær algebra er å

\begin{itemize}
\item løse likninger
\item løse likningssystemer
\end{itemize}

\subsection{Likningssystemer}

\begin{equation*}
\begin{alignedat}{10}
	&x_{1}	&+	&	2	&x_{2}	&-	&	3	&x_{3}	&	\quad	&=	\quad	&		&1	\\
2	&x_{1}	&-	&		&x_{2}	&+	&		&x_{3}	&	\quad	&=	\quad	&		&2	\\
-	&x_{1}	&+	&		&x_{2}	&-	&	5	&x_{3}	&	\quad	&=	\quad	&	-	&1
\end{alignedat}
\end{equation*}

Vi kan ofte se på et likningssystem hvordan det lar seg løse på forhånd. Likningssystemer med

\begin{itemize}
\item Færre likninger enn ukjente vil ha uendelig mange løsninger eller ingen.
\item Like mange likninger som ukjente vil som regel ha en løsning.
\item Flere likninger enn ukjente vil som regel ha ingen løsning.
\end{itemize}

Et likningssystem med flere likninger enn ukjente er nødt til å ha lineært uavhengige likninger for å kunne løses. For at et likningsystem skal ha en løsning må det være like mange lineært uavhengige likninger som ukjente. I tillegg kan man også finne likningssystemer som ikke lar seg løse fordi likningene motsier hverandre:

\begin{equation*}
\begin{alignedat}{5}
2x	&+	&		&	&y	&=	3	\\
4x	&+	&	2	&	&y	&=	4
\end{alignedat}
\end{equation*}

Her ser du et eksempel på en løsning med uendelig mange løsninger.
\begin{align*}
x + y &=5	&	(x,y) &= (t, 5-t)
\end{align*}

\newpage

\subsection{Matriser}
\subsubsection*{Regneregler}
For en matrise $A$ med dimensjoner $m \times n$
\begin{align*}
A+B &= B+A	&	A \cdot B &\neq B \cdot A	&	A(BC) &= (AB)C	&	\frac{A}{B} &= A \cdot B^{-1}	\\
B \cdot B^{-1} &= I_{n}	&	A \cdot I_{n} &=  A		&	I_{m} \cdot A &= A	&	(A^{T})^{T} &= A
\end{align*}

\subsubsection*{Identitetsmatrisen}
\begin{align*}
I_{2} = 
\begin{pmatrix}
1	&	0	\\
0	&	1
\end{pmatrix}
\end{align*}

\subsubsection*{Addisjon og substraksjon}
\begin{align*}
\begin{pmatrix}
1	&	2	\\
3	&	4	\\
5	&	6
\end{pmatrix}
+
\begin{pmatrix}
a	&	b	\\
c	&	d	\\
e	&	f
\end{pmatrix}
&=
\begin{pmatrix}
a	&	2b	\\
3c	&	4d	\\
5e	&	6f
\end{pmatrix}\\
(m \times n) + (m \times n) &= (m \times n)
\end{align*}

Begge matrisene må ha samme dimensjoner for å addere og subtrahere to matriser.

\subsubsection*{Multiplikasjon}
\begin{align*}
\begin{pmatrix}
a	&	b	&	c	\\
d	&	e	&	f	\\
g	&	h	&	i	\\
\end{pmatrix}
\cdot
\begin{pmatrix}
1	&	4	\\
2	&	5	\\
3	&	6	\\
\end{pmatrix}
&=
\begin{pmatrix}
a + 2b + 3c	&	4a + 5b + 6c	\\
d + 2e + 3f	&	4d + 5e + 6f	\\
g + 2h + 3i	&	4g + 5h + 6i
\end{pmatrix}	\\
(m \times n) \cdot (n \times p) &= (m \times p)
\end{align*}

For $A \cdot B$ må $A$ ha like mange kolonner som $B$ har rader. Resultanten vil da ha like mange rader som $A$ og kolonner som $B$.

\subsubsection*{Transponering}
\begin{align*}
\begin{pmatrix}
6	&	4	&	24	\\
1	&	-9	&	8
\end{pmatrix}^{T}
=
\begin{pmatrix}
6	&	1	\\
4	&	-9	\\
24	&	8
\end{pmatrix}
\end{align*}

\newpage

\subsubsection{Invers}
Hvis det finnes en matrise $B$ slik at $AB = BA = I_{n}$, da er $B$ inversmatrisen til $A$, $A^{-1}$. For å finne inversmatrisen $A^{-1}$ gjør man rekkeredusering av $A|I$ helt til man får flyttet identitetsmatrisen over på venstre side.
\begin{align*}
\begin{pmatrix}
a	&	b	&	1	&	0	\\
c	&	d	&	0	&	1
\end{pmatrix}
\rightarrow
\begin{pmatrix}
1	&	0	&	x	&	x	\\
0	&	1	&	x	&	x
\end{pmatrix}
\end{align*}

Hvis $A \cdot A^{-1} = I_{n}$ er den riktig.

\subsubsection{Determinant}
\begin{align*}
\det A =
\begin{vmatrix}
a	&	b	\\
c	&	d
\end{vmatrix}
= ad - bc
\end{align*}

\begin{align*}
\begin{vmatrix}
a	&	b	&	c	\\
1	&	2	&	3	\\
4	&	5	&	6
\end{vmatrix}
=
a \cdot 
\begin{vmatrix}
2	&	3	\\
5	&	6
\end{vmatrix}
- b \cdot
\begin{vmatrix}
1	&	3	\\
4	&	6
\end{vmatrix}
+ c \cdot
\begin{vmatrix}
1	&	2	\\
4	&	5
\end{vmatrix}
\end{align*}

\subsubsection{Egenverdier}
Hvis det finnes en vektor $\vec{v} \in \mathbb{R}^{n} \neq 0$, slik at $A \vec{v} = \lambda \vec{v}$, så er $\lambda$ egenverdien til $A$. Vi kan bruke det karakteristiske polynomet $\det(A - I\lambda)$ for å finne egenverdien $\lambda$. Ved å regne ut 

\subsubsection{Egenvektorer}
Vi kan finne egenvektorene til matrisen $A$ ved å regne ut likningssystemet $\vec{v} \cdot \vec{x} = \vec{0}$ for hver egenverdi.

\subsubsection{Eksponensiell}
Vi kan finne $A^{n}$ for matrisen $A$ med uttrykket $A_{n}=PD^{n}P^{-1}$ der $P$ er egenvektorene og $D$ er egenverdiene diagonalt.





\newpage
\section{Komplekse tall}
Komplekse tall skrives på formen $z = x + iy$, der $i$ er et imaginært tall.
\begin{align*}
i &= \sqrt{-1}	&	i^2 &= -1	&	i^3 &= i^2 \cdot i = -1 \cdot i = -i	&	i^4 &= i^2 \cdot i^2 = -1 \cdot -1 = 1
\end{align*}

\paragraph*{Eksempel}
Regn ut $z_{1} \cdot z_{2}$ der $z_{1} = 1 + 3i$ og $z_{2} = 2 - i$

\begin{align*}
z_{1} \cdot z_{2} &= (1 + 3i) \cdot (2 - i)	\\
&= 2 - i + 6i - 3i^2	\\
&= 2 + 5i -3 \cdot (-1)	\\
&= 5 + 5i
\end{align*}

\paragraph*{Eksempel}
Regn ut $\frac{z_{1}}{z_{2}}$ der $z_{1} = 2-3i$ og $z_{2} = 3+i$

\begin{align*}
\frac{z_{1}}{z_{2}}	=	\frac{2 - 3i}{3 + i}	=	\frac{(2 - 3i) \cdot (3 - i)}{(3 + i) \cdot (3-i)}	=	\frac{6 - 2i - 9i + 3i^2}{9 - i^2}	=	\frac{3}{10} - \frac{11}{10}i
\end{align*}

\paragraph*{Eksempel}
Løs likningen $2z + 3 -3i = \bar{z} - 3z - 1 + 3i$

\begin{align*}
2z + 3 -3i &= \bar{z} - 3z - 1 + 3i	\\
5z - \bar{z} &= -4 + 6i
\end{align*}

Så setter vi inn $z = x + iy$ og $\bar{z} = x - iy$

\begin{align*}
5(x + iy) - (x -iy) &= -4 + 6i	\\
5x + 5iy - x + iy &= -4 + 6i	\\
4x + 6iy &= -4 + 6i
\end{align*}

\begin{align*}
&	\text{Realdelen}		&	4x &= -4	&	&\rightarrow	&	x &= -1	\\
&	\text{Imaginærdelen}	&	6iy &= 6i	&	&\rightarrow	&	y &= 1
\end{align*}

\begin{align*}
z = x + iy = -1 + i
\end{align*}






\end{document}
